%
% link_budget.tex
%
% Copyright (C) 2021 by SpaceLab.
%
% OBDH 2.0 Documentation
%
% This work is licensed under the Creative Commons Attribution-ShareAlike 4.0
% International License. To view a copy of this license,
% visit http://creativecommons.org/licenses/by-sa/4.0/.
%

%
% \brief Test report of the v0.5 hardware.
%
% \author Gabriel Mariano Marcelino <gabriel.mm8@gmail.com>
%
% \institution Universidade Federal de Santa Catarina (UFSC)
%
% \version 0.6.0
%
% \date 2021/04/17
%

\chapter{Test Report of v0.5 Version} \label{anx:test-report-v05}

This appendix is a test report of the first manufactured and assembled PCB (version v0.5).

\begin{itemize}
    \item PCB manufacturer: PCBWay (China)
    \item PCB assembly: PCBWay (China)
    \item PCB arrival date: 2021/04/14
    \item Test execution date: 2021/04/16 to \textcolor{red}{TBC}
    \item Tester: G. M. Marcelino
\end{itemize}

\section{Visual Inspection}

The visual inspection is the first test executed when a new board is assembled. Performing this test on the v0.5 boards, no problems were found. The top view of the board can be seen in \autoref{fig:obdh2-v05-top}, the bottom view is available in \autoref{fig:obdh2-v05-bottom}.

\begin{figure}[!ht]
    \begin{center}
        \includegraphics[width=\columnwidth]{figures/v05/obdh2-v05-top.jpg}
        \caption{Top view of the OBDH 2.0 v0.5 board.}
        \label{fig:obdh2-v05-top}
    \end{center}
\end{figure}

\begin{figure}[!ht]
    \begin{center}
        \includegraphics[width=\columnwidth]{figures/v05/obdh2-v05-bottom.jpg}
        \caption{Bottom view of the OBDH 2.0 v0.5 board.}
        \label{fig:obdh2-v05-bottom}
    \end{center}
\end{figure}

\section{Firmware Programming}

\begin{figure}[!ht]
    \begin{center}
        \includegraphics[width=0.8\columnwidth]{figures/v05/log-first-boot.png}
        \caption{Log messages during the first boot.}
        \label{fig:log-first-boot}
    \end{center}
\end{figure}

\section{Communication Busses}

Material:

\begin{itemize}
    \item Saleae Logic Analyzer (24 MHz, 8 channels)
    \item Saleae Logic software (v1.2.18)
    \item MSP-FET Flash Emulation Tool
\end{itemize}

\subsection{I2C Port 0}

\begin{figure}[!ht]
    \begin{center}
        \includegraphics[width=0.7\columnwidth]{figures/v05/test-i2c-0.jpg}
        \caption{Connections of the I2C port 0 test.}
        \label{fig:test-i2c-0}
    \end{center}
\end{figure}

\begin{figure}[!ht]
    \begin{center}
        \includegraphics[width=\columnwidth]{figures/v05/waveform-i2c-0.png}
        \caption{Waveform of the I2C port 0.}
        \label{fig:waveform-i2c-0}
    \end{center}
\end{figure}

\subsection{I2C Port 1}

\begin{figure}[!ht]
    \begin{center}
        \includegraphics[width=0.7\columnwidth]{figures/v05/test-i2c-1.jpg}
        \caption{Connections of the I2C port 1 test.}
        \label{fig:test-i2c-1}
    \end{center}
\end{figure}

\begin{figure}[!ht]
    \begin{center}
        \includegraphics[width=\columnwidth]{figures/v05/waveform-i2c-1.png}
        \caption{Waveform of the I2C port 1.}
        \label{fig:waveform-i2c-1}
    \end{center}
\end{figure}

\subsection{I2C Port 2}

\begin{figure}[!ht]
    \begin{center}
        \includegraphics[width=0.7\columnwidth]{figures/v05/test-i2c-2.jpg}
        \caption{Connections of the I2C port 2 test.}
        \label{fig:test-i2c-2}
    \end{center}
\end{figure}

\begin{figure}[!ht]
    \begin{center}
        \includegraphics[width=\columnwidth]{figures/v05/waveform-i2c-2.png}
        \caption{Waveform of the I2C port 2.}
        \label{fig:waveform-i2c-2}
    \end{center}
\end{figure}

\section{Sensors}

\subsection{Input Voltage}

.

\subsection{Input Current}

\begin{figure}[!htb]
    \begin{center}
        \subfigure[Current sensing circuit.\label{fig:current-sensing-circuit-v05}]{\includegraphics[width=0.8\textwidth]{figures/v05/current-sensor-circuit.png}}

        \subfigure[MAX9934 pinout.\label{fig:max9934-pinout}]{\includegraphics[width=0.4\textwidth]{figures/v05/max9934-top-view.png}}
        ~
        \subfigure[Current sensing layout (bottom layer).\label{fig:}]{\includegraphics[width=0.4\textwidth]{figures/v05/current-sensor-layout.png}}
        \caption{.}
        \label{fig:current-sensing-error-v05}
    \end{center}
\end{figure}

\begin{figure}[!ht]
    \begin{center}
        \includegraphics[width=0.8\columnwidth]{figures/v05/log-current-sensor.png}
        \caption{Log messages with the read values from the current sensor.}
        \label{fig:log-current-sensor}
    \end{center}
\end{figure}

\subsection{Temperature}

.

\section{Peripherals}

\subsection{NOR Flash Memory}

\begin{figure}[!ht]
    \begin{center}
        \includegraphics[width=0.7\columnwidth]{figures/v05/test-nor-memory.jpg}
        \caption{Connections of the NOR flash memory test.}
        \label{fig:test-nor-memory}
    \end{center}
\end{figure}

\begin{figure}[!ht]
    \begin{center}
        \includegraphics[width=\columnwidth]{figures/v05/waveform-spi-mem.png}
        \caption{Waveform of the NOR memory SPI.}
        \label{fig:waveform-spi-mem}
    \end{center}
\end{figure}

\section{Conclusion}

.
